\documentclass{proposalnsf}
% This class file has been tweaked to death by LBarba to fit precisely the 
% formatting strictures of NSF, while still being rather pretty.

%%--------------------------------------------------------------------  PROCESS WITH XeLaTeX
%\usepackage{fontspec}% provides font selecting commands 
%\usepackage{paralist}       % compactitem environment
%\usepackage{xunicode}% provides unicode character macros 
%\usepackage{xltxtra} % provides some fixes/extras 
%\setromanfont[Mapping=tex-text,
%                 SmallCapsFont={Palatino},
%                 SmallCapsFeatures={Scale=0.85}]{Palatino}
%\setsansfont[Scale=0.85]{Trebuchet MS} 
%\setmonofont[Scale=0.85]{Monaco}

\renewcommand{\captionlabelfont}{\bf\sffamily}
\usepackage[hang,flushmargin]{footmisc} 
% 'hang' flushes the footnote marker to the left,  'flushmargin'  flushes the text as well.

% Define the color to use in links:
\definecolor{linkcol}{rgb}{0.459,0.071,0.294}
\definecolor{sectcol}{rgb}{0.63,0.16,0.16} % {0,0,0}
\definecolor{propcol}{rgb}{0.75,0.0,0.04}

\definecolor{gray}{rgb}{0.25,0.25,0.25}
\definecolor{ngreen}{rgb}{0.7,0.7,0.7} % a darker shade of green

\usepackage[
    %xetex,
    pdftitle={NSF proposal},
    pdfauthor={Rachel Slaybaugh, Kaitlin Thaney, Lorena Barba, C. Titus Brown, Paul Wilson, Ethan White, Tracy Teal, Greg Wilson, and Kathryn Huff},
    pdfpagemode={UseOutlines},
    pdfpagelayout={TwoColumnRight},
    bookmarks, bookmarksopen,bookmarksnumbered={True},
    pdfstartview={FitH},
    colorlinks, linkcolor={sectcol},citecolor={sectcol},urlcolor={sectcol}
    ]{hyperref}

%% Define a new style for the url package that will use a smaller font.
\makeatletter
\def\url@leostyle{%
  \@ifundefined{selectfont}{\def\UrlFont{\sf}}{\def\UrlFont{\small\ttfamily}}}
\makeatother
%% Now actually use the newly defined style.
\urlstyle{leo}


% this handles hanging indents for publications
\def\rrr#1\\{\par
\medskip\hbox{\vbox{\parindent=2em\hsize=6.12in
\hangindent=4em\hangafter=1#1}}}


\addto\captionsamerican{%
  \renewcommand{\refname}%
    {References Cited}%
} % solution found here: http://www.tex.ac.uk/cgi-bin/texfaq2html?label=latexwords

\def\baselinestretch{1}
\setlength{\parindent}{0mm} \setlength{\parskip}{0.8em}

\newlength{\up}
\setlength{\up}{-4mm}

\newlength{\hup}
\setlength{\hup}{-2mm}

\sectionfont{\large\bfseries\color{sectcol}\vspace{-2mm}}
\subsectionfont{\normalsize\it\bfseries\vspace{-4mm}}
\subsubsectionfont{\normalsize\mdseries\itshape\vspace{-4mm}} %\itshape
\paragraphfont{\bfseries}

%\usepackage[top=0.75in, bottom=0.75in, left=1in, right=1in]{geometry}
%\pagestyle{empty}
%\usepackage{tabu}

\begin{document}

\begin{center}
\large\textbf{Facilities, Equipment and Other Resources}
\end{center}

\textbf{University of California, Berkeley}

In addition to the workshop resources listed below, the following
facilities and equipment will be provided by the University of
California, Berkeley (UCB):

\begin{compactitem}

\item
  The facility and audio-visual (A/V) equipment for the annual meeting
  of project participants.

\item
  Meeting space for UCB's the Hacker Within group.

\end{compactitem}

Beyond these direct resources, UCB hosts many organizations that will
provide natural points of collaboration for this project, such as
Berkeley Institute for Data Science (BIDS), D-Lab, and Berkeley
Initiative for Transparency in the Social Sciences (BITTS).

D-Lab has sent a letter of support offering:

\begin{compactitem}

\item
  Space, outreach, and assistance in locating and supporting the
  assessment position.

\item
  D-Lab's platform to recruit other partners, particularly in
  developing and implementing assessment strategies and tools, and to
  offer public presentations or trainings.

\item
  To develop further concrete ways to support this proposal's
  trainings and assessments.

\end{compactitem}

%--------------------------------------------------------------
\textbf{Workshop Locations}

All locations hosting workshops, which includes the five universities
participating in this proposal and the additional sites that will be
identified, will provide the facilities and equipment for hosting the
workshops:

\begin{compactitem}

\item
  Presentation room(s) of sufficient size to host the workshop (at
  least 40 students).

\item
  A/V equipment for hosting the workshop.

\item
  A local point of contact to facilitate workshop organization.

\end{compactitem}

%--------------------------------------------------------------
\textbf{Other Participating Universities}

In addition to the workshop resources listed above, the four partner
universities (George Washington University, Michigan State University,
University of Wisconsin -- Madison and Utah State University) will
provide:

\begin{compactitem}

\item
  Project administrative support.

\item
  Meeting space for the local Hacker Within-style group.

\end{compactitem}

%--------------------------------------------------------------
\textbf{Mozilla}

The Mozilla Foundation will provide or arrange for presentation room(s)
of sufficient size to host one workshop annually for female students in
science and engineering, and another for students who are members of
underrepresented minorities in STEM disciplines.

\end{document}
